\documentclass[a4paper,11pt]{article}
\usepackage[a4paper, margin=20mm]{geometry}
\usepackage[T1]{fontenc}
\usepackage[utf8]{inputenc}
\usepackage{mathfunc}

% Header
% ===========================

\title{Integral Domains}
\author{John B. Fraleigh, A First Course In Abstract Algebra (Chapter 19)}
\date{May 27, 2015}


% Document
% ===========================

\begin{document}
\maketitle
\pagenumbering{gobble}

\begin{outline}

  \dbullet{19.1}
    If \(a\) and \(b\) are two nonzero elements of a ring \(R\) such that \(ab = 0\), then \(a\) and 
    \(b\) are "diviors of \(0\)" (or "\(0\) divisors").

  \dbullet{19.2}
    In the ring \(\mathbb{Z}_n\), the divisors of \(0\) are precisely those nonzero elements that are 
    not relatively prime to n.
    
    \begin{proof}
      Let \(m \in \mathbb{Z}_n\), where \(m \neq 0\), and let the gcd of \(m\) and \(n\) be \(d \neq 1\). Then
      \(m(\frac{n}{d}) = (\frac{m}{d})n\), and \((\frac{m}{d})n\) gives \(0\) as a multiple of \(n\). Thus 
      \(m(\frac{n}{d}) = 0\) in \(\mathbb{Z}_n\), while neither \(m\) nor \(\frac{n}{d}\) is \(0\), so \(m\) 
      is a \(0\) divisor. On the other hand, suppose \(m \in \mathbb{Z}_n\) is relatively prime to \(n\). If 
      for \(x \in \mathbb{Z}_n\) we have \(ms = 0\), then \(n\) divides the product \(ms\) of \(m\) and \(s\) 
      as elements in the ring \(\mathbb{Z}\). Since \(n\) is relatively prime to \(m\), \(n\) must divide \(s\), 
      so \(s = 0\) in \(\mathbb{Z}_n\).
    \end{proof}
      
  \cbullet{19.3}
    If \(p\) is a prime, then \(\mathbb{Z}_p\) has no divisors of \(0\).
      
  \tbullet{19.4}
    The cancellation laws hold in a ring \(R\) if and only if \(R\) has no \(0\) divisors.
    
    \begin{proof}
      \forward
        Let \(R\) be a ring in which the cancellation laws hold. suppose \(ab = 0\) for some \(a, b \in R\).
        We must show either \(a\) or \(b\) is \(0\). If \(a \neq 0\), then \(ab = a0\) implies \(b = 0\) by cancellation laws.
        Similarly, if \(b \neq 0\), \(ab = 0b\) implies \(a = 0\).
        
      \backward
        Suppose \(R\) has no divisors of \(0\). Suppose \(ab = ac\) with \(a \neq 0\). Then  \(ab - ac = a(b-c) = 0\). 
        Since \(a \neq 0\), and since \(R\) has no divisors of \(0\), we must have \(b - c = 0\), so \(b = c\). 
        Similarly, \(ba = ca\) with \(a \neq 0 \Rightarrow (b - c)a = 0 \Rightarrow b - c = 0 \Rightarrow b = c\).
    \end{proof}

  \dbullet{19.5}
    An "integral domain" \(D\) is a commutative ring with unity \(1 \neq 0\) and containing no divisors of \(0\).
    
  \tbullet{19.6}
    Every field is an integral domain.
    
    \begin{proof}
      Let \(a, b \in F\), and suppose \(a \neq 0\). Then if \(ab = 0\), we have \((\frac{1}{a})(ab) = (\frac{1}{a})0 = 0\).
      But then \(0 = [(\frac{1}{a})]b = 1b = b\). Thus \(ab = 0\) with \(a \neq 0 \Rightarrow b = 0\), so \(F\) has no \(0\)
      divisors. \(F\) is clearly a commutative ring with unity.
    \end{proof}
    
  \tbullet{19.7}
    Every finte integral domain is a field.
    
    \begin{proof}
      Let \(0, 1, a_1, \ldots, a_n\) be all the elements of a finite domain \(D\). We must show for \(a \in D\), where
      \(a \neq 0, \exists b \in D\) such that \(ab = 1\). Consider \(a1, aa_1, \ldots, aa_n\). We claim these elements
      are distinct, since \(aa_i = aa_j \Rightarrow a_i = a_j\) by the cancellation laws. Since \(D\) has no \(0\) 
      divisors, none of these elements are \(0\). Hence, by counting, we find \(a1, aa_1, \ldots, aa_n\) is a permutation
      of \(1, a_1, \ldots, a_n\). Thus, either \(a1 = 1 (a = 1)\) or \(aa_i = 1\) for some \(i\). Therefore \(a\) has a
      multiplicative inverse.
    \end{proof}

  \cbullet{19.8}
    If \(p\) is a prime, then \(\mathbb{x}_p\) is a field.
    
    \begin{proof}
      \(\mathbb{Z}_p\) is an integral domain. Apply Theorem 19.7.
    \end{proof}
      
  \dbullet{19.9}
    If for a ring \(R\) a positive integer \(n\) exists such that \(n \cdot a = 0\) for all \(a \in R\),
    then the least such positive integer is the "characteristic of the ring \(R\)." If no such positive integer 
    exists, then \(R\) is of "characteristic 0."
      
  \tbullet{19.10}
    Let \(R\) be a ring with unity. If \(n \cdot 1 \neq 0\) for all \(n \in \mathbb{Z}^{+}\), then
    \(R\) has characteristic \(0\). If \(n \cdot 1 = 0\) for some \(n \in \mathbb{Z}^{+}\), then the smallest 
    such integer n is the characteristic of \(R\).
    
    \begin{proof}
      If \(n \cdot 1 \neq 0\) for all \(n \in \mathbb{Z}^{+}\), then surely we cannot have \(n \cdot a = 0\) for all 
      \(a \in R\) for some positive integer \(n\). Suppoes \(n\) is a positive integer such that \(n \cdot 1 = 0\). Then
      for any \(a \in R\), we have \(n \cdot a = a + a + \ldots + a = a(1 + 1 + \ldots + 1) = a(n \cdot 1) = a0 = 0\).
      The theorem follows directly.
    \end{proof}

\end{outline}

\end{document}
