\documentclass[a4paper,11pt]{article}
\usepackage[a4paper, margin=20mm]{geometry}
\usepackage[T1]{fontenc}
\usepackage[utf8]{inputenc}
\usepackage{mathfunc}

% Header
% ===========================

\title{Factor Groups}
\author{John B. Fraleigh, A First Course In Abstract Algebra (Chapter 14)}
\date{May 27\textsuperscript{th}, 2015}


% Document
% ===========================

\begin{document}
\maketitle
\pagenumbering{gobble}

\begin{outline}

  \tbullet{14.1}
    Let \(\phi: G \rightarrow G'\) be a group homomorphism with kernel \(H\). Then the cosets of \(H\) form a 
    "factor group," \(\sfrac{G}{H}\), where \((aH)(bH) = (ab)H\). Also, the map \(\mu: \sfrac{G}{H} \rightarrow
    \phi[G]\) defined by \(\mu(aH) = \phi(a)\) is an isomorphism. Both coset multiplication and \(\mu\) are 
    well-defined, independent of the choices of \(a\) and \(b\) from the cosets.
    
  \tbullet{14.2}
    Let \(H\) be a subgroup of a group \(G\). Then left coset multiplication is well defined by the
    equation \((aH)(bH) = (ab)H\) if and only if \(H\) is a normal subgroup.
    
    \begin{proof}
      \forward
        Suppose \((aH)(bH) = (ab)H\) is well defined on left cosets. Let \(a \in G\). We want to show
        \(aH = Ha\). Let \(x \in aH\). Choosing representatives \(x \in aH\) and \(a^{-1} \in a^{-1}H\), we 
        have \((xH)(a^{-1}H) = (xa^{-1})H\). On the other hand, choosing \(a \in aH\) and \(a^{-1} \in 
        a^{-1}H\) yields \((aH)(a^{-1}H) = H\). Since our operation is well defined, \(xa^{-1} = h \in H 
        \Rightarrow x = ha\), so \(x \in Ha\). Thus \(aH \subseteq Ha\). The symmetric proof is ommitted.
        
      \backward
        Now suppose \(H\) is a normal subgroup. Suppose we wish to computer \((aH)(bH)\). Then choosing
        \(a \in aH\) and \(b \in bH\), we obtain \((ab)H\). Next, choosing \(ah_{1} \in aH\) and \(bh_{2} 
        \in bH\), we obtain \((ah_{1}bh_{2})H\). Now \(h_{1}b \in Hb = bH\) so \(h_{1}b = bh_{3}\) for 
        some \(h_{3} \in H\). Thus, \((ah_{1})(bh_{2}) = a(h_{1}b)h_{2} = a(bh_{3})h_{2} = (ab)(h_{3}h_{2}) 
        \in (ab)H\).
    \end{proof}
    
  \cbullet{14.3}
    Let \(H\) be a normal subgroup of \(G\). Then the cosets of \(H\) form a group \(\sfrac{G}{H}\)
    under the binary operation \((aH)(bH) = (ab)H\).
    
  \dbullet{14.4}
    The group \(\sfrac{G}{H}\) in Corollary 14.3 is the "factor group" of \(G\) by \(H\).
    
  \tbullet{14.5}
    Let \(H\) be a normal subgroup of \(G\). Then \(\gamma: G \rightarrow \sfrac{G}{H}\) given by
    \(\gamma(x) = xH\) is a homomorphism with kernel \(H\).
    
    \begin{proof}
      Let \(x, y \in G\). Then \(\gamma(xy) = (xy)H = (xH)(yH) = \gamma(x)\gamma(y)\), so \(\gamma\) is indeed a 
      homomorphism. Since \(xH= H\) if and only if \(x \in H\), we see the kernel of \(\gamma\) is 
      indeed \(H\).
    \end{proof}
    
  \tbullet{14.6 (The Fundamental Homomorphism Theorem)}
    Let \(\phi: G \rightarrow G'\) be a group homomorphism with kernel \(H\). Then \(\phi[G]\) is a group, 
    and \(\mu: \sfrac{G}{H} \rightarrow \phi[G]\) given by \(\mu(gH) = \phi(g)\) is an isomorphism. If 
    \(\gamma: G \rightarrow \sfrac{G}{H}\) is the homomorphism given by \(\gamma(g) = gH\), then 
    \(\phi(g) = \mu\gamma(g)\) for each \(g \in G\).
    
  \tbullet{14.7}
    The following are three equivalent conditions for a subgroup \(H\) of a group \(G\) to be a normal subgroup of \(G\):
    \begin{enumerate}[i.]
      \item \(ghg^{-1} \in H\) for all \(g \in G\) and \(h \in H\).
      \item \(gHg^{-1} = H\) for all \(g \in G\).
      \item \(gH = Hg\) for all \(g \in G\).
    \end{enumerate}
    
  \dbullet{14.8}
    An isomorphism \(\phi: G \rightarrow G\) of a group \(G\) with itself is an "automorphism" of
    \(G\). The automorphism \(i_{g}: G \rightarrow G\) where \(i_{g}(x) = gxg^{-1}\) for all \(x \in G\), is the
    "inner automorphism of \(G\) by \(g\)." Performing \(i_{g}\) on \(x\) is called "conjugation of \(x\) by \(g\)."
    
\end{outline}

\end{document}
