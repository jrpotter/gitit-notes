\documentclass[a4paper,11pt]{article}
\usepackage[a4paper, margin=20mm]{geometry}
\usepackage[T1]{fontenc}
\usepackage[utf8]{inputenc}
\usepackage{mathfunc}

% Header
% ===========================

\title{Orthogonality and Least Squares}
\author{Linear Algebra With Applications by Otto Bretscher (Chapter 5)}
\date{June 10\textsuperscript{th}, 2015}


% Document
% ===========================

\begin{document}
\maketitle
\pagenumbering{gobble}

\begin{outline}

  \dbullet{5.1.1 (Orthogonality, Length, Unit Vectors)}
    \begin{enumerate}[i.]
      \item
        Two vectors \(\vec{v}\) and \(\vec{w}\) in \(\bbr^n\) are called "perpendicular" or "orthogonal" 
        if \(\dotp{v}{w} = 0\). 
      \item 
        The "length" (or "magnitude" or "norm") of a vector \(\vec{v}\) in \(\bbr^n\) is 
        \(\norm{\vec{v}} = \sqrt{\dotp{v}{v}}\). 
      \item
        A vector \(\vec{u}\) in \(\bbr^n\) is called a "unit vector" if its length is \(1\).
    \end{enumerate}
    
  \dbullet{5.1.2 (Orthogonal to Subspaces)}
    A vector \(\vec{x}\) in \(\bbr^n\) is said to be orthogonal to a subspace \(V\) of \(\bbr^n\) if 
    \(\vec{x}\) is orthogonal to all the vectors \(\vec{v}\) in \(V\), meaning that \(\dotp{x}{v} = 0\) for 
    all vectors \(\vec{v}\) in \(V\).
    
  \dbullet{5.1.3 (Orthonormal Vectors)}
    The vectors \(\inflatedot{\vec{u}}{m}\) in \(\bbr^n\) are called "orthonormal" if they are all unit vectors and orthogonal to one another:
    \[
      \vec{u}_i \cdot \vec{u}_j = \begin{cases}
        1\text{ if }i = j \\
        0\text{ if }i \neq j
      \end{cases}
    \]
    
  \tbullet{5.1.4 (Properties of Orthonormal Vectors)}
    \begin{enumerate}[i.]
      \item Orthonormal vectors are linearly independent.
      \item Orthonormal vectors \(\inflatedot{\vec{u}}{n}\) in \(\bbr^n\) form a basis of \(\bbr^n\).
    \end{enumerate}
    
    \begin{proof}
      \begin{enumerate}[i.]
        \item
          Consider a relation \[ c_1\vec{u}_1 + c_2\vec{u}_2 + \cdots + c_i\vec{u}_i + \cdots + c_m\vec{u}_m = \vec{0} \] 
          among the orthonormal vectors \(\inflatedot{\vec{u}}{m}\) in \(\bbr^n\). Let us form the dot product of each side 
          of this equation with \(\vec{u}_i\): 
          \[ 
            (c_1\vec{u}_1+c_2\vec{u}_2+\cdots+c_i\vec{u}_i+\cdots+c_m\vec{u}_m)\cdot\vec{u}_i = \vec{0}\cdot\vec{u}_i = 0
            \text{.} 
          \] 
          Because the dot product is distributive, 
          \[
            c_1(\vec{u}_1\cdot\vec{u}_i) + c_2(\vec{u}_2\cdot\vec{u}_i) + \cdots + c_i(\vec{u}_i\cdot\vec{u}_i) + 
            \cdots+c_m(\vec{u}_m\cdot\vec{u}_i) = 0\text{.} 
          \]
          We know that \(\vec{u}_i\cdot\vec{u}_i = 1\), and all other dot products are zero. Therefore \(c_i = 0\). 
          Since this holds for all \(i = 1, \ldots, m\), it follows that the vectors \(\inflatedot{\vec{u}}{m}\) are 
          linearly independent.
        \item
          From (\romannumeral 1), we note that the vectors \(\inflatedot{\vec{u}}{n}\) are linearly independent. We
          also note that any \(n\) linearly independent vectors in \(\bbr^n\) form a basis of \(\bbr^n\). 
          The theorem follows.
      \end{enumerate}
    \end{proof}

\end{outline}

\end{document}
