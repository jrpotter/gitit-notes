\documentclass[a4paper,11pt]{article}
\usepackage[a4paper, margin=20mm]{geometry}
\usepackage[T1]{fontenc}
\usepackage[utf8]{inputenc}
\usepackage{mathfunc}

% Header
% ===========================

\title{Direct Products and Finitely Generated Abelian Groups}
\author{A First Course In Abstract Algebra by John B. Fraleigh (Chapter 11)}
\date{May 27\textsuperscript{th}, 2015}


% Document
% ===========================

\begin{document}
\maketitle
\pagenumbering{gobble}

\begin{outline}

  \tbullet{11.1}
    Let \(G_{1}, G_{2}, \ldots, G_{n}\) be groups. For \((a_{1}, a_{2}, \ldots, a_{n})\) and
    \((b_{1}, b_{2}, \ldots, b_{n})\) in \(\prod_{i=1}^{n} G_{i}\), define \((a_{1}, a_{2}, \ldots, 
    a_{n})(b_{1},  b_{2}, \ldots, b_{n})\) to be the element \((a_{1}b_{1}, a_{2}b_{2}, \ldots, a_{n}b_{n})\). Then 
    \(\prod_{i=1}^{n} G_{i}\) is a group, the "direct product of the groups \(G_{i}\)," under this binary operation.
    
  \tbullet{11.2}
    The group \(\mathbb{Z}_{m} \times \mathbb{Z}_{n}\) is cyclic and is isomorphic to 
    \(\mathbb{Z}_{mn}\) if and only if \(m\) and \(n\) are relatively prime, that is, the gcd of \(m\) and \(n\)
    is \(1\).
    
    \begin{proof}
      \forward 
        Consider the cyclic subgroup of \(\mathbb{Z}_{m} \times \mathbb{Z}_{n}\) generated 
        by \((1, 1)\). Under addition by components, the first component \(1 \in \mathbb{Z}_{m}\) yields \(0\) 
        after \(m\) summands, and the second \(1 \in \mathbb{Z}_{n}\) after \(n\) summands. For them to 
        yield \(0\) simultaneously, the summands must be a multiple of \(m\) and \(n\). Thus 
        \(\mathbb{Z}_{m} \times \mathbb{Z}_{n}\) is cyclic of order \(mn\) and isomorphic to \(\mathbb{Z}_{mn}\).
        
      \backward 
        Suppose \(gcd(m, n) = d > 1\). Then \(\frac{mn}{d}\) is divisible by both \(m\) and \(n\). 
        Consequently, for any \((r, s)\) in \(\mathbb{Z}_{m} \times \mathbb{Z}_{n}\), \(\frac{mn}{d}\) summands
        of \((r, s)\) is \((0, 0)\). Hence no element \((r, s) \in \mathbb{Z}_{m} \times \mathbb{Z}_{n}\) can
        generate the entire group and \(\mathbb{Z}_{m} \times \mathbb{Z}_{n}\) is not cyclic and consequently not
        isomorphic to \(\mathbb{Z}_{mn}\).
        
    \end{proof}
    
  \cbullet{11.3}
    The group \(\prod_{i=1}^{n} \mathbb{Z}_{m_{i}}\) is cyclic and isomorphic to \(\mathbb{Z}_{m_{1}, 
    m_{2}, \ldots, m_{n}}\) if and only if the numbers \(m_{i}\) for \(i = 1, \ldots, n\) are such that 
    the gcd of any two of them is \(1\).
    
  \dbullet{11.4}
    Let \(r_{1}, r_{2}, \ldots, r_{n}\) be positive integers. Their "least common multiple,"
    abbreviated lcm, is the positive generator of the cyclic group of all common multiples of the \(r_{i}\).
    
  \tbullet{11.5}
    Let \((a_{1}, a_{2}, \ldots, a_{n}) \in \prod_{i=1}^{n} G_{i}\). If \(a_{i}\) is of
    finite order \(r_{i}\) in \(G_{i}\), then the order of \((a_{1}, a_{2}, \ldots, a_{n}\) in 
    \(\prod_{i=1}^{n} G_{i}\) is equal to the least common multiple of all the \(r_{i}\).
    
  \tbullet{11.6 (Fundamental Theorem of Finitely Generated Abelian Groups)}
    Every finitely generated abelian group \(G\) is isomorphic to a direct product of cyclic groups in 
    the form \(\mathbb{Z}_{(p_{1})^{r_{1}}} \times \mathbb{Z}_{(p_{2})^{r_{2}}} \times \ldots \times 
    \mathbb{Z}_{(p_{n})^{r_{n}}} \times \mathbb{Z} \times \ldots \times \mathbb{Z}\), where the \(p_{i}\) 
    are primes, not necessarily distinct, and the \(r_{i}\) are positive integers. The direct product is 
    unique except for possible rearragement of the factors.
    
  \dbullet{11.7}
    A group \(G\) is "decomposable" if it is isomorphic to a direct product of two proper
    nontrivial subgroups. Otherwise \(G\) is "indecomposable."
    
  \tbullet{11.8}
    The finite indecomposable abelian groups are exactly the cyclic groups with order a power of a prime.
    
    \begin{proof}
      \forward 
        Let \(G\) be a finite indecomposable abelian group. Then \(G\) is isomorphic to a 
        direct product of cyclic groups of prime power order. Since G is indecomposable, this direct product 
        must consist of just one cyclic group whose order is a power of a prime number.
        
      \backward 
        Conversely, let \(p\) be a prime. Then \(\mathbb{Z}_{p^{r}}\) is indecomposable, for 
        if \(\mathbb{Z}_{p^{r}}\) were isomorphic to \(\mathbb{Z}_{p^{i}} \times \mathbb{Z}_{p^{j}}\), where 
        \(i + j = r\), then every element would have an order at most \(p^{max(i, j)} < p^{r}\).
    \end{proof}
    
  \tbullet{11.9}
    If \(m\) divides the order of a fiite abelian group \(G\), then \(G\) has a subgroup of order \(m\).
    
    \begin{proof}
      We can think of \(G\) as being \(\mathbb{Z}_{(p_{1})^{r_{1}}} \times \mathbb{Z}_{(p_{2})^{r_{2}}} \times 
      \ldots \times \mathbb{Z}_{(p_{n})^{r_{n}}}\), where note all primes \(p_{i}\) need to be distinct. Since
      \((p_{1})^{r_{1}} (p_{2})^{r_{2}} \ldots (p_{n})^{r_{n}}\) is the order of \(G\), then \(m\) must be of the
      form \((p_{1})^{s_{1}} (p_{2})^{s_{2}} \ldots (p_{n})^{s_{n}}\), where \(0 \leq s_{i} \leq r_{i}\). Note
      \((p_{i})^{r_{i}-s_{i}}\) generates a cyclic subgroup of \(\mathbb{Z}_{(p_{i})^{r_{i}}}\) of order
      \(\frac{(p_{i})^{r_{i}}}{gcd((p_{i})^{r_{i}}, (p_{i})^{r_{i}-s_{i}})}\). Also note 
      \(gcd((p_{i})^{r_{i}}, (p_{i})^{r_{i}-s_{i}})\) is just \((p_{i})^{r_{i}-s_{i}}\) so the order is 
      \((p_{i})^{s_{i}}\). Thus \(\langle (p_{1})^{r_{1}-s_{1}} \rangle \times \langle (p_{2})^{r_{2}-s_{2}} 
      \rangle \times \ldots \times \langle (p_{n})^{r_{n}-s_{n}} \rangle\) is the required subgroup.
    \end{proof}
    
  \tbullet{11.10}
    If \(m\) is a square free integer, that is, \(m\) is not divisible by the square of 
    any prime, then every abelian group of order \(m\) is cyclic.
    
    \begin{proof}
      Let \(G\) be an abelian group of square free order \(m\). So \(G\) is isomorphic to 
      \(\mathbb{Z}_{(p_{1})^{r_{1}}} \times \mathbb{Z}_{(p_{2})^{r_{2}}} \times \ldots \times 
      \mathbb{Z}_{(p_{n})^{r_{n}}}\) where m = \((p_{1})^{r_{1}} (p_{2})^{r_{2}} \ldots (p_{n})^{r_{n}}\).
      Since \(m\) is square free, all \(r_{i} = 1\) and all \(p_{i}\) are distinct primes. So \(G\) is
      isomorphic to \(\mathbb{Z}_{p_{1} p_{2} \ldots p_{n}}\), and \(G\) is therefore cyclic.
    \end{proof}

\end{outline}

\end{document}
