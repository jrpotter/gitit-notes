\documentclass[a4paper,11pt]{article}
\usepackage[a4paper, margin=20mm]{geometry}
\usepackage[T1]{fontenc}
\usepackage[utf8]{inputenc}
\usepackage{mathfunc}

% Header
% ===========================

\title{Random Variables}
\author{A First Course in Probability by Sheldon Ross (Chapter 4)}
\date{June 09\textsuperscript{th}, 2015}


% Document
% ===========================

\begin{document}
\maketitle
\pagenumbering{gobble}

\begin{outline}

  \dbullet{4.1.1 (Cumulative Distribution Function)}
    For a random variable \(X\), the function \(F\) defined by \[F(x)=\bbp[X\leq x], -\infty < x < \infty\]
    is called the "cumulative distributive function," or, more simply, the "distribution function," of \(X\). 
    Thus, the distribution function specifies, for all real values \(x\), the probabiliity that the random 
    variable is less than or equal to \(x\).
    
  \dbullet{4.2.1 (Discrete Random Variables)}
    A random variable that can take on at most a countable number of possible values is said to be "discrete."
    
  \dbullet{4.2.2 (Probability Mass Function)}
    For a discrete random variable \(X\), we define the "probability mass function" \(p(a)\) of \(X\) by 
    \[p(a) = \bbp[X=a]\text{.}\] The probability mass function \(p(a)\) is positive for at most a countable 
    number of values of \(a\). That is, if \(X\) must assume one of the values \(x_1, x_2, \ldots,\) then 
    \begin{align*}
      p(x_i) &\geq 0\text{ for }i = 1, 2, \ldots \\
      p(x) &= 0\text{ for all other values of }x
    \end{align*}
    
  \dbullet{4.3.1 (Expected Value)}
    If \(X\) is a discrete random variable having a probability mass function \(p(x)\), then the "expectation," or the "expected value," of \(X\), denoted by \(E[X]\), is defined by \[E[X] = \sum_{x:p(x)>0}xp(x)\text{.}\]
    
  \dbullet{4.3.2 (Indicator Variables)}
    We say that \(I\) is an "indicator variable" for the event \(A\) if 
    \[ I = \begin{cases} 
             1 \text{ if } A \text{ occurs} \\
             0 \text{ if } A^c \text{ occurs}
       \end{cases}
    \]

\end{outline}

\end{document}
