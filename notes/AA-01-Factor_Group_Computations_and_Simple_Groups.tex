\documentclass[a4paper,11pt]{article}
\usepackage[a4paper, margin=20mm]{geometry}
\usepackage[T1]{fontenc}
\usepackage[utf8]{inputenc}
\usepackage{mathfunc}

% Header
% ===========================

\title{Factor Group Computations and Simple Groups}
\author{A First Course In Abstract Algebra by John B. Fraleigh (Chapter 15)}
\date{May 27\textsuperscript{th}, 2015}


% Document
% ===========================

\begin{document}
\maketitle
\pagenumbering{gobble}

\begin{outline}

  \tbullet{15.1}
    Let \(G = H \times K\) be the direct product of groups \(H\) and \(K\). Then \(\bar{H} = \{(h, e) : h \in H\}\) 
    is a normal subgroup of \(G\). Also \(\sfrac{G}{\bar{H}}\) is isomorphic to \(K\) in a natural way. Similarly, 
    \(\sfrac{G}{\bar{K}} \simeq H\) in a natural way.
    
    \begin{proof}
      Consider the homomorphism \(\pi_{2}: H \times K \rightarrow K\), where \(\pi_{2}(h, k) = k\). Because
      \(Ker(\pi_{2}) = \bar{H}\), then \(\bar{H}\) is a normal subgroup of \(H \times K\). Because 
      \(\pi_{2}\) is onto \(K\), \(\sfrac{(H \times K)}{\bar{H}} \simeq K\).
    \end{proof}

  \tbullet{15.2}
    A factor group of a cyclic group is cyclic.
    
    \begin{proof}
      Let \(G\) be cyclic with gernator \(a\), and let \(N\) be a normal subgroup of \(G\). We claim the coset \(aN\)
      generates \(\sfrac{G}{N}\). We must compute all powers of \(aN\). But this amounts to computing all powers of 
      \(a \in G\) which yields \(G\). Hence \(\langle aN \rangle\) gives all coset of \(N\) and \(\sfrac{G}{N}\) is cyclic.
    \end{proof}
    
  \dbullet{15.3}
    A group is "simple" if it is nontrivial and has no nontrivial proper subgroups.
    
  \tbullet{15.4}
    The alternating group \(A_{n}\) is simple for \(n \geq 5\).
    
  \tbullet{15.5}
    Let \(\phi: G \rightarrow G'\) be a group homomorphism. If \(N\) is a normal subgroup of \(G\), then 
    \(\phi[N]\) is a normal subgroup of \(\phi[G]\). Also, if \(N'\) is a normal subgroup of \(\phi[G]\),
    then \(\phi^{-1}[N']\) is a normal subgroup of \(G\).
    
  \dbullet{15.6}
    A "maximal normal subgroup of a group \(G\)" is a normal subgroup \(M\) not equal to \(G\) such that 
    there is no proper normal subgroup \(N\) of \(G\) properly containing \(M\).
    
  \tbullet{15.7}
    \(M\) is a maximal normal subroup of \(G\) if and only if \(\sfrac{G}{M}\) is simple.
    
    \begin{proof}
      \forward 
        Let \(M\) be a maximal normal subgroup of \(G\). Consider the canonical homomorphism
        \(\gamma: G \rightarrow \sfrac{G}{M}\). Now \(\gamma^{-1}\) of any nontrivial proper normal 
        subgroup of \(\sfrac{G}{M}\) is a proper normal subgroup of \(G\) properly containing \(M\). 
        But \(M\) is maximal, so this cannot happen, meaning \(\sfrac{G}{M}\) is simple.
        
      \backward
        If \(N\) is a normal subgroup of \(G\) properly containing \(M\), then \(\gamma[N]\) is normal is 
        \(\sfrac{G}{M}\). If also \(N \neq G\), then \(\gamma[N] \neq \sfrac{G}{M}\) and \(\gamma[N] \neq
        \{M\}\). Since \(\sfrac{G}{M}\) is simple, no such \(\gamma[N]\) or \(N\) can exist, and \(M\) is maximal.
    \end{proof}
    
  \tbullet{15.8}
    Let \(G\) be a group. The set of all commutators \(aba^{-1}b^{-1}\) for \(a, b \in G\)
    generates a subgroup \(C\) (the "commutator subgroup") of \(G\). This subgroup \(C\) is a normal subgroup of
    \(G\). Furthermore, if \(N\) is a normal subgroup of \(G\), then \(\sfrac{G}{N}\) is abelian if and only if
    \(C \leq N\).
    
    \begin{proof}
      The commutators certainly generate a subgroup \(C\). Note the inverse \((aba^{-1}b^{-1})\) of a commutator is 
      a commutator, namely \(bab^{-1}a^{-1}\). Also, \(e = eee^{-1}e^{-1}\) is a commutator. Then \(C\) consists 
      precisely of all finite products of commutators. 
      
      For \(x \in C\), we must show that \(g^{-1}xg \in C\) for all \(g \in G\). By
      inserting \(e = gg^{-1}\) between each product of commutators occurring in \(x\), we see it is sufficient to show 
      that for each commutator \(cdc^{-1}d^{-1}\) that \(g(cdc^{-1}d^{-1})g^{-1}\) is in \(C\). But:
      \begin{align*}
        g^{-1}(cdc^{-1}d^{-1})g &= (g^{-1}cdc^{-1})(e)(d^{-1}g) \\
                                &= (g^{-1}cdc^{-1})(gd^{-1}dg^{-1})(d^{-1}g)\\
                                &= [(g^{-1}c)d(g^{-1}c)^{-1}d^{-1}][dg^{-1}d^{-1}g],
      \end{align*}
      which is in \(C\). So \(C\) is normal. 
      
      That \(\sfrac{G}{C}\) is abelian follows from \((aC)(bC) = abC = ab(b^{-1}a^{-1}ba)C\) which equals 
      \(abb^{-1}a^{-1})baC = baC = (bC)(aC).\) 
      
      Furthermore, if \(N\) is a normal subgroup of \(G\) and \(\sfrac{G}{N}\) is abelian, then 
      \((a^{-1}N)(b^{-1}N) = (b^{-1}N)(a^{-1}N)\); that is, \(aba^{-1}b^{-1}N = N\), so \(aba^{-1}b^{-1} \in N\), 
      and \(C \leq N\). Finally, if \(C \leq N\), then \((aN)(bN) = abN = ab(b^{-1}a^{-1}ba)N\) which equals 
      \((abb^{-1}a^{-1})baN = baN = (bN)(aN).\)
    \end{proof}
      
\end{outline}

\end{document}
