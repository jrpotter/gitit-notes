\documentclass[a4paper,8pt]{article}
\usepackage[a4paper, margin=15mm]{geometry}
\usepackage[T1]{fontenc}
\usepackage[utf8]{inputenc}
\usepackage{mathfunc}

% Header
% ===========================

\title{Finite Automata}
\author{Elements of the Theory of Computation by H.R. Lewis and C.H. Papadimitrious (Chapter 2)}
\date{June 15\textsuperscript{th}, 2015}


% Document
% ===========================

\begin{document}
\maketitle
\pagenumbering{gobble}

\begin{outline}

  \dbullet{2.1.1 (Deterministic Finite Automaton)}
    A "deterministic finite automaton" is a quintuple \(M=(K,\Sigma,\delta,s,F)\) where

    \quad\(K\) is a finite set of "states,"

    \quad\(\Sigma\) is an alphabet,

    \quad\(s \in K\) is the "initial state,"

    \quad\(F \subseteq K\) is the set of "final states," and

    \quad\(\delta\), the "transition function," is a function from \(K \times \Sigma\) to \(K\).

  \dbullet{2.2.1 (Nondeterministic Finite Automaton)}
    A "nondeterministic finite automaton" is a quintuple \(M = (K, \Sigma, \Delta, s, F)\), where

    \quad\(K\) is a finite set of "states,"

    \quad\(\Sigma\) is an alphabet,

    \quad\(s \in K\) is the "initial state,"

    \quad\(F \subseteq K\) is the set of "final states," and

    \quad\(\Delta\), the "transition relation," is a subset of \(K\times(\Sigma\cup\{e\})\times K\).

  \tbullet{2.2.3 (Automaton Equivalence)}
    For each nondeterministic finite automaton, there is an equivalent deterministic finite automaton.

    \begin{proof}
      Let \(M=(K,\Sigma,\Delta,s,F)\) be a nondeterministic finite automaton. We shall construct a deterministic
      finite automaton \(M'=(K',\Sigma,\delta,s',F')\) equivalent to \(M\). First we define the "epsilon closure" as follows:

      For any state \(q \in K\), let \(E(q)\) be the set of all states of \(M\) that are reachable from state \(q\) without
      reading any input. That is, \[ E(q) = \{p\in K: (q,e) \vdash_M^* (p, e)\}\text{.} \]

      Define deterministic automaton \(M'=(K',\Sigma,\delta,s',F')\) such that \(K'=2^K, s'=E(s),F=\{Q \subseteq K: Q \cap F \neq \emptyset\}\),
      and for each \(Q \subseteq K\) and each symbol \(a \in \Sigma\), define
      \[ \delta'(Q, a) = \bigcup\{E(p): p \in K \text{ and } (q,a,p) \in \Delta \text{ for some } q \in Q\}\text{.} \]
      We note this formalization shows \(M'\) is deterministic, since \(\delta'\) is single-valued and well defined on all \(Q \in K'\)
      and \(a \in \Sigma\), by construction.

      We now \textit{claim} that for any string \(w \in \Sigma^*\) and any states \(p, q \in K'\),
      \[ (q,w) \vdash_M^* (p, e) \text{ if and only if } (E(q), w) \vdash_{M'}^* (P, e) \] for some set \(P\) containing \(p\). Given this,
      for any string \(w \in \Sigma^*\), \(w \in L(M)\) if and only if \((s, w)\vdash_M^*(f, e)\) for some \(f \in F\) if and only if
      \((E(s), w) \vdash_{M'}^* (Q, e)\) for some \(Q\) containing \(f\) meaning \(w \in L(M')\), proving the theorem.

      We now prove the claim by induction on \(|w|\).

      \textit{Basis Step}: For \(|w| = 0\), we must show \((q, e) \vdash_M^* (p, e)\) if and only if \((E(q), e) \vdash_{M'}^* (P, e)\) for
      some set \(P\) containing \(p\). But note this states that if \(q = p\), then \(p \in P\) since \(E(q) = P\). This is true by hypothesis,
      so the basis step holds.

      \textit{Induction Hypothesis}: The claim is true for all strings \(w\) of length \(k\) or less for some \(k \geq 0\).

      \textit{Induction Step}:

      \forward Suppose \((q,w)\vdash_M^*(p,e)\). Then there are states \(r_1\) and \(r_2\) such that
      \[ (q,w) \vdash_M^* (r_1, a) \vdash_M (r_2, e) \vdash_M^* (p, e)\text{.} \]
      where \(w = va\) for some input \(v\) and \(a\). Now \((q, va) \vdash_M^* (r_1, a)\) is tantamount to \((q, v)\vdash_M^*(r_1, e)\), and
      since \(|v| = k\), the induction hyposthesis states \((E(q), v) \vdash_{M'}^* (R_1, e)\) for some set \(R_1\) containing \(r_1\). Since
      \((r_1, a) \vdash_M (r_2, e)\), there is a triple \((r_1, a, r_2) \in \Delta\), and hence \(E(r_2) \subseteq \delta'(R_1, a)\), by
      construction of \(M'\). Since \((r_2, e) \vdash_M^* (p, e)\), it follows that \(p \in E(r_2)\), and therefore \(p \in \delta'(r_1, a)\).
      Therefore \((R_1, a) \vdash_{M'} (P, e)\) for some \(P\) containing \(p\), and thus \((E(q), va) \vdash_{M'}^* (R_1, a) \vdash_{M'} (P, e)\).

      \backward Suppose that \((E(q), va) \vdash_{M'}^* (R_1, a) \vdash_{M'} (P, e)\) for some \(P\) containing \(p\) and some \(R_1\) such that
      \(\delta'(R_1, a) = P\). Since \(p \in P = \delta'(R_1, a)\), there is some \(r_2\) such that \(p \in E(r_2)\), and, for some \(r_1 \in R_1\),
      \((r_1, a, r_2)\) is a transition of \(M\). Thus, by the induction hypothesis, \((q, v) \vdash_M^* (r_1, a) \vdash_M (r_2, e) \vdash_M^* (p, e)\),
      where the last relation holds by the epsilon closure of \(r_2\).
    \end{proof}

  \tbullet{2.3.1 (Thompson's Construction Algorithm)}
    The class of languages accepted by finite automata is closed under:
    \begin{enumerate}[i.]
      \item union
      \item concatenation
      \item Kleene star
      \item complementation
      \item intersection
    \end{enumerate}

  \tbullet{2.3.2 (Regular Language and Finite Automata)}
    A language is regular if and only if it is accepted by a finite automaton.

    \begin{proof}
      \forward Note the class of regular languages is the closure of the empty set \(\emptyset\) and singletons \(a\) under union,
      concatenation, and Kleene star. By Thompson's Construction Algorithm (a constructive proof), we note this must hold.

      \backward Let \(M = (K, \Sigma, \Delta, s, F)\) be a finite automaton, not necessarily deterministic. We shall construct a regular
      expression \(R\) such that \(L(R) = L(M)\). Let \(K = \{\inflatedot{q}{n}\}\) and \(s = q_1\). For \(i, j = 1, 2, \ldots, n\) and
      \(k = 0, 1, \ldots, n\), we define \(R(i, j, k)\) as the set of all strings in \(\Sigma^*\) that may drive \(M\) from state
      \(q_i\) to \(q_j\) without passing through any intermediate state numbered \(k+1\) or greater. Therefore, when \(k=n\),
      \[ R(i,j,n) = \{ w \in \Sigma^*: (q_i, x) \vdash_M^* (q_j, e) \}\text{.} \] Therefore \[ L(M) = \bigcup\{R(1,j,n): q_j \in F\}\text{.} \]
      We note that each of these sets \(R(i,j,k)\) are regular by the following inductive argument on \(k\):

      \textit{Basis:} For \(k=0\), \(R(i,j,0)\) is either \(\{a \in \Sigma \;\cup\: \{e\}: (q_i, a, q_j) \in \Delta\}\) if \(i \neq j\), or it is
      \(\{e\} \:\cup\: \{a \in \Sigma \cup \{e\}: (q_i, a, q_j) \in \Delta\}\) if \(i = j\). Each of these sets if finite and therefore regular.

      \textit{Induction Hypothesis:} For all \(k \geq 0\), \(R(i, j, k)\) is regular for all \(i, j\).

      \textit{Induction Step:} Consider set \(R(i,j,k+1)\). Note it can be defined as follows
      \[ R(i,j,k+1) = R(i,j,k) \cup R(i,k+1,k)R(k+1,k+1,k)^*R(k+1,j,k)\text{.} \]
      Since each \(R(i,j,k)\) is regular, so must \(L(M)\), since it is the union of a finite number of regular languages.
    \end{proof}

\end{outline}

\end{document}
