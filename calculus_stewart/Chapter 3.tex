\documentclass[a4paper,11pt]{article}
\usepackage[a4paper, margin=20mm]{geometry}
\usepackage[T1]{fontenc}
\usepackage[utf8]{inputenc}
\usepackage{mathfunc}

% Header
% ===========================

\title{Differentiation Rules}
\author{James Stewart, Calculus Early Transcendentals (Chapter 2)}
\date{May 30\textsuperscript{th}, 2015}


% Document
% ===========================

\begin{document}
\maketitle
\pagenumbering{gobble}

\begin{outline}

  \tbullet{3.1.1}
    Let \((f(x)=c\) for some constant \(c\). Then \(\leibniz{x}(c) = 0\).
    
    \begin{proof}
      Note the following:
      \begin{align*}
        f'(x) &= \xlimit{h}{0}\frac{f(x+h)-f(x)}{h}\\
              &= \xlimit{h}{0}\frac{c-c}{h}        \\
              &= \xlimit{h}{0}0                    \\
              &= 0\text{.}
      \end{align*}
    \end{proof}
    
  \tbullet{3.1.2}
    If \(c\) is a constant and \(f\) a differentiable function, then \(\leibniz{x}[cf(x)] = c\leibniz{x}f(x)\).
    
    \begin{proof}
      Let \(g(x) = cf(x)\). Then 
      \begin{align*}
        \xlimit{h}{0} \frac{g(x+h)-g(x)}{h} &= \xlimit{h}{0} c\frac{f(x+h)-f(x)}{h}\\
                                            &= c \xlimit{h}{0}\frac{f(x+h)-f(x)}{h}\\
                                            &= cf'(x)
      \end{align*}
    \end{proof}
    
  \tbullet{3.1.3}
    If \(f, g\) are differentiable, then \(\leibniz{x}[f(x)+g(x)] = \leibniz{x}f(x) + \leibniz{x}g(x)\).
    
    \begin{proof}
      Let \(F(x) = f(x) + g(x)\). Then:
      \begin{align*}
        F'(x) &= \xlimit{h}{0}\frac{F(x+h)-F(x)}{h}\\
              &= \xlimit{h}{0}\frac{f(x+h)+g(x+h)-f(x)-g(x)}{h}\\
              &= \xlimit{h}{0}\frac{f(x+h)-f(x)}{h} + \xlimit{h}{0}\frac{g(x+h)-g(x)}{h}\\
              &= \leibniz{x}f(x) + \leibniz{x}g(x)\text{.}
      \end{align*}
    \end{proof}
    
  \dbullet{3.1.4}
    \(e\) is the number such that \(\xlimit{h}{0}\frac{e^h-1}{h} = 1\).
    
  \tbullet{3.1.5}
    \(\leibniz{x}(e^x) = e^x\).
    
    \begin{proof}
      Let \(f(x) = a^x\) for some base \(a\). Then:
      \begin{align*}
        f'(x) &= \xlimit{h}{0}\frac{f(x+h)-f(x)}{h} \\
              &= \xlimit{h}{0}\frac{a^{x+h}-a^x}{h} \\
              &= \xlimit{h}{0}a^x\left(\frac{a^h-1}{h}\right)  \\
              &= a^x\xlimit{h}{0}\frac{a^h-1}{h}\text{.}
      \end{align*}
      Therefore \(f'(e) = e^x\xlimit{h}{0}\frac{e^h-1}{h}\), which, by Definition 3.1.4, is just \(e^x\).
    \end{proof}

  \tbullet{3.2.1}
    If \(f\) and \(g\) are both differentiable, the \(\leibniz{x}[f(x)g(x)] = f(x)\leibniz{x}g(x) + g(x)\leibniz{x}f(x)\).
    
    \begin{proof}
      Let \(u = f(x), v = g(x)\), and \(\Delta x\) represent a change in \(x\). Then \(\Delta(uv) = (\dinc{u})
      (v + \Delta v) - uv = u\Delta v + v\Delta u + \Delta u\Delta v\). Then dividing by \(\Delta x\), we get:
      \begin{align*}
        \leibniz{x}(uv) &= \frac{\Delta(uv)}{\Delta x}\\
                        &= \xlimit{\Delta x}{0}\left[u\frac{\Delta v}{\Delta x} + v\frac{\Delta u}{\Delta x} + \Delta u
                            \frac{\Delta v}{\Delta x}\right] \\
                        &= u\left(\xlimit{\Delta x}{0}\frac{\Delta v}{\Delta x}\right) + 
                           v\left(\xlimit{\Delta x}{0}\frac{\Delta v}{\Delta x}\right) + 
                           \left(\xlimit{\Delta x}{0}\Delta u\right)
                           \left(\xlimit{\Delta x}{0}\frac{\Delta v}{\Delta x}\right)\\
                        &= u\leibniz{x}(v) + v\leibniz{x}(u) + 0 \cdot \leibniz{x}(v)\\
                        &= f(x)\leibniz{x}[g(x)] + g(x)\leibniz{x}[f(x)]\text{.}
      \end{align*}
    \end{proof}
    
  \tbullet{3.2.2}
    \(\leibniz{x}[f/g] = [f'g-fg']/(g^2)\).
    
    \begin{proof}
      Note the following:
      \begin{align*}
        \Delta \left(\frac{u}{v}\right) 
          &=\frac{\dinc{u}}{\dinc{v}} - \frac{u}{v}\\
          &=\frac{(\dinc{u})v-(\dinc{v})u}{v(\dinc{v})}\\
          &=\frac{v\Delta u-u\Delta v}{v(\dinc{v})}\\
      \end{align*}
      which implies
      \begin{align*}
        \leibniz{x}\left(\frac{u}{v}\right) 
          &= \xlimit{x}{0}\frac{\Delta(\sfrac{u}{v})}{\Delta x}\\
          &= \xlimit{x}{0}\frac{v\left(\frac{\Delta u}{\Delta x}\right) -
             u\left(\frac{\Delta v}{\Delta x}\right)}{v(\dinc{v})}\text{.}
      \end{align*}
      Therefore
      \[
        \leibniz{x}\left(\frac{u}{v}\right) 
          = \frac{v\xlimit{x}{0}\frac{\Delta u}{\Delta x}-u\xlimit{x}{0}\frac{\Delta v}{\Delta x}}{v\xlimit{x}{0}(\dinc{v})}
      \]
      where \(\xlimit{x}{0} \Delta v = 0\) since \(v\) is differentiable and therefore continuous. Hence, 
      \[
        \leibniz{x}\left(\frac{u}{v}\right) = \frac{f'g-fg'}{g^2}\text{.}
      \]
    \end{proof}
    
  \tbullet{3.3.1}
    \(\leibniz{x}(\sin{x}) = \cos{x}\).
    
    \begin{proof}
      Let \(f(x) = \sin{x}\). Then:
      \begin{align*}
        f'(x) &= \xlimit{h}{0}\frac{\sin{(x+h)}-\sin{x}}{h}\\
              &= \xlimit{h}{0}\frac{\sin{x}\cos{h}+\cos{x}\sin{h}-\sin{h}}{h}\\
              &= \xlimit{h}{0}\sin{x}\cdot\xlimit{h}{0}\frac{\cos{h}-1}{h} +
                 \xlimit{h}{0}\cos{x}\cdot\xlimit{h}{0}\frac{\sin{h}}{h}
      \end{align*}
      In addition, 
      \begin{gather*}
        \sin{\theta}<\theta\text{ for }0<\theta<\frac{\pi}{2}\Rightarrow\frac{\sin{\theta}}{\theta}<1\\
        \theta\leq\tan{\theta}\Rightarrow\theta<\frac{\sin{\theta}}{\cos{\theta}}
          \Rightarrow\cos{\theta}<\frac{\sin{\theta}}{\theta}<1\text{.}
      \end{gather*}
      Then by the Squeeze Theorem, \(\xlimit{\theta}{0} = 1\). 
      
      Also,
      \begin{align*}
        \xlimit{\theta}{0}\frac{\cos{\theta}-1}{\theta}
          &= \xlimit{\theta}{0}\left(\frac{\cos{\theta}-1}{\theta}\cdot\frac{\cos{\theta}+1}{\cos{\theta}+1}\right)\\
          &= \xlimit{\theta}{0}\frac{\cos^2\theta-1}{\theta(\cos{\theta}+1}\\
          &= \xlimit{\theta}{0}\frac{-\sin^2\theta}{\theta(\cos{\theta}+1)}\\
          &= -\xlimit{\theta}{0}\frac{\sin{\theta}}{\theta}\cdot\xlimit{\theta}{0}\frac{\sin{\theta}}{\cos{\theta}+1}\\
          &= -1\cdot\frac{0}{1+1}\\
          &= 0
      \end{align*}
      Thus, \(f'(x) = \sin{x}\cdot 0 + \cos{x}\cdot 1 = \cos{x}\).
    \end{proof}
    
  \tbullet{3.3.2}
    \(\leibniz{x}(\cos{x}) = -\sin{x}\).
    
  \tbullet{3.3.3}
    \begin{enumerate}[i.]
      \item \(\leibniz{x}(\tan{x})=\sec^2{x}\)
      \item \(\leibniz{x}(\cot{x})=-\csc^2{x}\)
      \item \(\leibniz{x}(\csc{x})=-\csc{x}\cot{x}\)
      \item \(\leibniz{x}(\sec{x})=\sec{x}\tan{x}\)
    \end{enumerate}
    
    \begin{proof}
      Repeated applications of the Quotient Rule.
    \end{proof}
    
  \tbullet{3.4.1}
    If \(g\) is differentiable at \(x\) and \(f\) is differentiable at \(g(x)\), then the composite
    function \(F=f\circ g\) defined by \(F(x)=f(g(x))\) is differentiable at \(x\) and \(F'\) is given
    by the product \(F'(x)=f'(g(x))\cdot g'(x)\).
    
    \begin{proof}
      Suppose \(u=g(x)\) is differentiable at \(a\) and \(y=f(u)\) is differentiable at \(b=g(a)\). Let
      \(\Delta x, \Delta u\), and \(\Delta y\) be an increment of \(x, u\), and \(y\) respectively. As an aside,
      if \(y_1=f(x_1)\) and \(x_1\) changes from \(a_1\) to \(a_1 + \Delta x_1\), then 
      \begin{gather*}
        \Delta y_1 = f(a_1+\Delta x_1)-f(a_1)\text{ and }\\
        f'(a_1) = \xlimit{x_1}{0}\frac{\Delta y_1}{\Delta x_1}
      \end{gather*}
      If we let 
      \[
        \epsilon = \frac{\Delta y_1}{\Delta x_1} - f'(a_1)\text{, then }
        \xlimit{x_1}{0}\epsilon = f'(a_1)-f'(a_1) = 0
      \]
      Also note 
      \[
        \epsilon = \frac{\Delta y_1}{\Delta x_1} - f'(a_1) \Rightarrow \Delta y_1=f'(a_1)\Delta x_1 + 
        \epsilon\Delta x_1\text{ where } \epsilon \rightarrow 0\text{ as }\Delta x_1 \rightarrow 0
      \]
      Now, we can say
      \begin{gather*}
        \Delta u=g'(a)\Delta x + \epsilon_1\Delta x = [g'(a)+\epsilon_1]\Delta x\text{ where } \epsilon_1
          \rightarrow 0\text{ as }\Delta x\rightarrow 0\text{,}\\
        \Delta y=f'(b)\Delta u + \epsilon_2\Delta u = [f'(b)+\epsilon_2]\Delta u\text{ where } \epsilon_2 
          \rightarrow 0\text{ as }\Delta u\rightarrow 0\text{.}
      \end{gather*}
      Substituting, we find \(\Delta y=[f'(b)+\epsilon_2][g'(a)+\epsilon_1]\Delta x\) which implies
      \(\frac{\Delta y}{\Delta x} = [f'(b)+\epsilon_2][g'(a)+\epsilon_1]\). As \(\Delta x\rightarrow 0\), 
      \(\Delta u\rightarrow 0\) which implies both \(\epsilon_1\rightarrow 0\) and \(\epsilon_2\rightarrow 0\)
      as \(\Delta x\rightarrow 0\). 
      Therefore 
      \[
        \leibniz[y]{x} = \xlimit{x}{0}[f'(b)+\epsilon_2][g'(a)+\epsilon_1] = f'(b)g'(a) = f'(g(a))g'(a)\text{.}
      \]
    \end{proof}
  
  \tbullet{3.4.2}
    \(\leibniz{x}(a^x) = a^x\ln{a}\).
    
    \begin{proof}
      Note \(a = e^{\ln{a}}\) so \(a^x = e^{(\ln{a})x}\). Then \(\leibniz{x}(a^x) = \leibniz{x}(e^{(\ln{a})x})
      = e^{(\ln{a})x} \cdot \leibniz{x}[(\ln{a})x]\) by the Chain Rule. This in turn equals \(a^x\cdot \ln{a}\).
    \end{proof}
    
  \tbullet{3.5.1}
    \(\leibniz{x}(\sin^{-1} x) = \frac{1}{\sqrt{1-x^2}}\).
    
    \begin{proof}
      Note \(y = \sin^{-1}x \Rightarrow \sin{y}=x\). Then:
      \[
        \leibniz{x}(\sin{y}) = \leibniz{x}(x) \Rightarrow \cos{y}\leibniz[y]{x} = 1 
        \Rightarrow \leibniz[y]{x} = \frac{1}{\cos{y}}. 
      \]
      Now \(\cos{y}\geq 0\), since \(\frac{-\pi}{2} \leq y \leq \frac{\pi}{2}\), so \(\cos{y} = \sqrt{1-\sin^2{y}} 
      = \sqrt{1-x^2}\). 
      
      Therefore \[ \leibniz[y]{x} = \frac{1}{\cos{y}} = \frac{1}{\sqrt{1-x^2}}\text{.} \]
    \end{proof}
    
  \tbullet{3.5.2}
    \begin{enumerate}[i.]
      \item \(\leibniz{x}(\cos^{-1}x) = -\frac{1}{\sqrt{1-x^2}}\)
      \item \(\leibniz{x}(\tan^{-1}x) = \frac{1}{1+x^2}\)
      \item \(\leibniz{x}(\csc^{-1}x) = -\frac{1}{x\sqrt{x^2-1}}\)
      \item \(\leibniz{x}(\sec^{-1}x) - \frac{1}{x\sqrt{x^2-1}}\)
      \item \(\leibniz{x}(\cot^{-1}x) = =\frac{1}{1+x^2}\)
    \end{enumerate}
    
    \begin{proof}
      Use of implicit differentiation.
    \end{proof}
    
  \tbullet{3.6.1}
    \(\leibniz{x}(\log_{a}x) = \frac{1}{x\ln{a}}\).
    
    \begin{proof}
      Let \(y = \log_{a}x \Rightarrow a^y = x\). Thus:
      \begin{align*}
                    &\leibniz{x}(a^y) = \leibniz{x}(x)\\
        \Rightarrow\; &a^y(\ln{a})\leibniz[y]{x} = 1\\
        \Rightarrow\; &\leibniz[y]{x} = \frac{1}{a^y\ln{a}} = \frac{1}{x\ln{a}}\text{.}
      \end{align*}
    \end{proof}
    
  \cbullet{3.6.2}
    \(\leibniz{x}(\ln{x}) = \frac{1}{x}\).
    
    \begin{proof}
      \(\leibniz{x}(\ln{x}) = \leibniz{x}(\log_{e}x) = \frac{1}{x\ln{e}} = \frac{1}{x}\).
    \end{proof}
    
  \tbullet{3.6.3}
    If \(n\) is any real number and \(f(x) = x^n\), then \(f'(x)=nx^{n-1}\).
    
    \begin{proof}
      We let \(y = x^n\) and use logarithmic differentiation. \(y = x^n \Rightarrow |y| = |x^n| \Rightarrow
      \ln{|y|} = \ln{|x^n|} = n\ln{|x|}\). Therefore \(\leibniz{x}(\ln{|y|}) = \leibniz{x}(n\ln{|x|}) \Rightarrow
      \frac{y'}{y} = \frac{n}{x} \Rightarrow \leibniz[y]{x} = \frac{yn}{x} = \frac{nx^n}{x} = nx^{n-1}\).
    \end{proof}
    
  \tbullet{3.6.3}
    \(e = \xlimit{x}{0}(1+x)^{\frac{1}{x}}\).
    
    \begin{proof}
      If \(f(x) = \ln{x}\), then \(f'(x) = \frac{1}{x} \Rightarrow f'(1) = 1\). Now \(f'(1) = \xlimit{h}{0}
      \frac{f(1+h)-f(1)}{h} = \xlimit{x}{0}\frac{\ln{(1+x)}-\ln{1}}{x} = \xlimit{x}{0}\frac{1}{x}\ln{(1+x)}
      = \xlimit{x}{0} \ln{(1+x)}^{\frac{1}{x}}\). Then \(e = e^{1} = \exp{(\xlimit{x}{0}\ln{(1+x)}^{\frac{1}{x}})}
      = \xlimit{x}{0} \exp(\ln({1+x})^{\frac{1}{x}}) = \xlimit{x}{0}(1+x)^{\frac{1}{x}}\).
    \end{proof}
    
  \tbullet{3.8.1}
    If \(y(t)\) is the value of a quantity \(y\) at time \(t\), and if the rate of change of \(y\) with respect
    to \(t\) is proportional to its size \(y(t)\) at any time, then \(\leibniz[y]{t}=ky\).
    
  \tbullet{3.8.2}
    The only solutions of the differential equation \(\leibniz[y]{t} = ky\) are the exponential functions
    \(y(t)=y(0)e^{kt}\).
    
  \dbullet{3.10.1}
    The "linear approximation" or "tangent line approximation" of \(f\) at \(a\) is \(f(x) \approx f(a) + f'(a)(x-a)\).
    
  \dbullet{3.10.2}
    The linear function whose graph is the tangent line \(L(x)=f(a)+f'(a)(x-a)\) is called the "linearization"
    of \(f\) at \(a\).
    
  \dbullet{3.10.3}
    Let \(y = f(x)\), where \(f\) is a differentiable function. Then the "differential" \(dx\) is an independent
    variable and differential \(dy\) is deifned in terms of \(dx\) as \(dy = f'(x)dx\).
    
  \dbullet{3.11.1}
    The hyperbolic functions are:
    \begin{enumerate}[i.]
      \item \(\sinh{x} = \frac{e^x - e^{-x}}{2}\)
      \item \(\cosh{x} = \frac{e^x+e^{-x}}{2}\)
      \item \(\tanh{x} = \frac{\sinh{x}}{\cosh{x}}\)
      \item \(\text{csch }x = \frac{1}{\sinh{x}}\)
      \item \(\text{sech }x = \frac{1}{\cosh{x}}\)
      \item \(\coth{x} = \frac{\cosh{x}}{\sinh{x}}\)
    \end{enumerate}
  
\end{outline}

\end{document}
