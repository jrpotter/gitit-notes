\documentclass[a4paper,11pt]{article}
\usepackage[a4paper, margin=20mm]{geometry}
\usepackage[T1]{fontenc}
\usepackage[utf8]{inputenc}
\usepackage{mathfunc}

% Header
% ===========================

\title{Combinatorial Analysis}
\author{Sheldon Ross, A First Course in Probability (Chapter 1)}
\date{May 29\textsuperscript{th}, 2015}


% Document
% ===========================

\begin{document}
\maketitle
\pagenumbering{gobble}

\begin{outline}

  \tbullet{1.4.1}
    \(\binom{n}{r} = \binom{n-1}{r-1} + \binom{n-1}{r}\) for \(1 \leq r \leq n\)
    
    \begin{proof}
      Consider a group of \(n\) objects and label one object 'object 1.' Now, there are \(\binom{n-1}{r-1}\)
      groups of size \(r\) that contain 'object 1,' since each group is formed by selecting \(r - 1\) from the remaining
      \(n - 1\) objects. Additionally, there are \(\binom{n-1}{r}\) groups of size \(r\) that do not contain
      'object 1.' Since a total of \(\binom{n}{r}\) groups of size \(r\), the theorem follows.
    \end{proof}

  \tbullet{1.4.2 (Binomial Theorem)}
    \((x+y)^n = \sum_{k=0}^n \binom{n}{k}x^ky^{n-k}\)
    
    \begin{proof}
      Consider the product \((x_1 + y_1)(x_2 + y_2)\ldots(x_n + y_n)\). Its expansion consists of the sum of \(2^n\)
      terms will contain as a factor either \(x_i\) or \(y_i\) for each \(i=1,2,\ldots,n\). Now we find how many of
      the \(2^n\) terms will have \(k\) of the \(x_i\)'s and \((n-k)\) of the \(y_i\)'s as factors. As each term
      consisting of \(k\) of the \(x_i\)'s and \((n-k)\) of the \(y_i\)'s corresponds to a choice of a group of \(k\)
      from the \(n\) values \(\inflatedot{x}{n}\), there are \(\binom{n}{k}\) such terms. Thus, letting \(x_i = x,
      y_i = y\), and \(i = 1, \ldots, n\), we see that \((x+y)^n = \sum_{k=0}^n \binom{n}{k}x^ky^k\).
    \end{proof}
    
  \dbullet{1.5.1}
    If \(\inflatedot{n}{r} = n\), we define \(\binom{n}{\inflatedot{n}{r}}\) by \(\frac{n!}{n_1!n_2!\ldots n_r!}\).
    Thus, \(\binom{n}{\inflatedot{n}{r}}\) represents the number of possible divisions of \(n\) distinct objects
    into \(r\) distinct groups of respective sizes \(\inflatedot{n}{r}\).
    
  \tbullet{1.5.2 (Multinomial Theorem)}
    \((\inflatedot{x}{r})^n\) is the sum of all nonnegative integer-valued vectors \((\inflatedot[+]{n}{r})\) such 
    that \(\inflatedot[+]{n}{r} = n\); that is:
    \[ 
      (\inflatedot[+]{x}{r})^n = \mathop{\sum_{(\inflatedot{n}{r}):}}_{\inflatedot[+]{n}{r}=n}
      \binom{n}{\inflatedot{n}{r}}x_1^{n_1}x_2^{n_2}\ldots x_r^{n_r}\text{.}
    \]
    
  \pbullet{1.6.1}
    There are \(\binom{n-1}{r-1}\) dinstinct positive integer-valued vectors \((\inflatedot{x}{r})\) satisfying
    the equation \(\inflatedot[+]{x}{r} = n, x_i > 0\) for \(i = 1, \ldots, r\).
    
  \pbullet{1.6.2}
    There are \(\binom{n+r-1}{r-1}\) distinct nonnegative integer-valued vectors \(\inflatedot{x}{r}\) satisfying
    the equation \(\inflatedot[+]{x}{r} = n\).
    
\end{outline}

\end{document}
